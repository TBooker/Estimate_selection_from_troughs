\begin{table}
   \centering
      \begin{threeparttable}[b]

\caption{Comparison of advantageous mutational models. Dashes indicate the best-fitting model. }

\begin{tabular}{ccccc}
\toprule
	& & & \multicolumn{2}{c}{$\Delta AIC$} \\
       Map & GC & Model\tnote{a} &  CNEs &  Exons \\
\midrule
 \multirow{9}{*}{\textit{castaneus}} &    \multirow{3}{*}{High} &     2 &      - &       - \\
  &      &     e &   -367 &    -284 \\
  &      &     s &    -85.6 &    -284 \\ \cdashline{2-5}
  &     \multirow{3}{*}{None}  &     2 &      - &       - \\
  &      &     e &   -262 &    -126.989377 \\
  &      &     s &    -87.3 &    -153 \\ \cdashline{2-5}
  &     \multirow{3}{*}{Paigen}  &     2 &      - &       - \\
  &      &     e &   -280 &    -134\\ 
  &      &     s &   -113 &    -161\\ \hdashline
\multirow{9}{*}{Cox}  &     \multirow{3}{*}{High} &     2 &      - &      -3.91\\
  &      &     e &   -177&       - \\
  &      &     s &    -92&      -1.48\\ \cdashline{2-5}
  &     \multirow{3}{*}{No} &     2 &      - &       - \\
  &      &     e &   -150&     -19.6\\
  &      &     s &    -44&     -38.9\\ \cdashline{2-5}
  &     \multirow{3}{*}{Paigen} &     2 &      - &       - \\
  &  	 &     e &   -147 &     -16.2\\
  &      &     s &    -43.5 &     -32.1\\
\bottomrule
\end{tabular}
 
   \begin{tablenotes}
     \item[a] Denotes the model of advantageous mutations used. $e$ - exponential, $2$ - two classes of discrete effects and $s$ - a single class of discrete effects.
   \end{tablenotes}

  \end{threeparttable}
  
  \label{tab:dDFE}
\end{table}