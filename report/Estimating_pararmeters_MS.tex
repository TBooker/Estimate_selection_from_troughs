\documentclass{article}
\usepackage{authblk}
\usepackage{arydshln}
\usepackage[margin=2cm]{geometry}
\usepackage{adjustbox}
\usepackage{multirow}
\usepackage{indentfirst}



\setlength{\parskip}{\baselineskip}%
\setlength{\parindent}{30pt}%

\title{\textbf{Estimating the parameters of selective sweeps from patterns of genetic diversity in the house mouse}}
\author[1,*]{Tom R. Booker}
\author[1]{Brian Charlesworth}
\author[1]{Peter D. Keightley}
\affil[1]{Institute of Evolutionary Biology, University of Edinburgh, Edinburgh}
\affil[*]{\emph{t.r.booker@sms.ed.ac.uk}}




\begin{document}
\maketitle
\begin{abstract}
Woah! Selective sweeps are neat. Let's look at a model for estimating their strength and frequency from pop. gen. data using a more developed version of the approach adopted by Wiehe and Stephan in 1993. Let's go on to compare the strength and frequnecy of new mutations occuring in protein-coding versus regulatory regions and see which one will contribute more to phenotypic change.

\end{abstract}

\section*{Introduction}

Random mutation is a fundamental force in evolution, it the ultimate source of all biodiversity. Populations may become better adapted to their environments (REF), experience 'mutational meltdown' (REF), undergo evolutionary rescue (REF) and become speciated all through the effects of new mutations (REF). In each of these processes, the fitness effects of the mutations in question is a fundamental parameter (REFS). In the case of adaptation, an enduring question has been whether mutations in regulatory or protein-coding portions of the genome have larger effects on fitness (King and Wilson etc.). Evidence in multiple species suggests that selection is frequent in non-coding portions of the mammalian genome (Halligan \textit{et al.} 2013; Booker and Keightley \textit{submitted}; MORE MORE MORE). With the 

In order to answer this question, however, methods are required for esitmating the rates of occurence and fitness effects of new mutations.

One class of methods for inferring the strength and frequnecies of new mutations relies on the  assumption that selected mutations segregate in natural populations. The distributon of allele frequencies in a sample, the site frequency spectrum, for a class of sites subject to natural selection 



In the past 30 years of population genetic research it has become clear that natural selection shapes patterns of genetic diversity across the genomes of many species (Corbett-Detig \textit{et al} 2015; Cutter and Payseur 2012). Because genetically linked sites in the genome do not evolve independantly, selection acting at one site may have consequences for linked sites. This process, which we shall term selection at linked sites, is intrinsically linked to the rate of recombiantion both by crossing-over and gene conversion (REF DUMP). 

One of the biggest goals of evolutionary genetics is to understand the fitness consequences of new mutations. There is undoubtedly a distribution of fitness effects for new mutations, but this distribution may depend on a number of factors such as the genomic region where the mutations occur and the fitness of the population in question (Peter's Review?). Experimental approaches for estimating the DFE are limited to organisms which can be maintained in lab populations, which typically excludes mammals. 

When advantageous mutations are driven to high frequencies by selection, they drag with them portions of the haplotype on which they are present. This process, termed a selective sweep, has been the subject of rigorous population genetic research (Maynard-Smith and Haigh, Hudson and Kaplan, Coop and Ralph, Hermisson and Pennings, Barton 2000). There are a number of models decribing the ways in which selective sweeps may proceed (reviewed in Booker \textit{et al.} 2017). One of the consequences of selective sweeps is a reduction in neutral diversity in the regions surrounding advantageous mutations. 

The time it takes an advantageous mutation to sweep through a population is proportional to the strength of selection and this has an effect on the level of neutral diversity in genomic regions linked to the mutation. As a mutation sweeps, linked neutral variants may get trapped by the sweep and carried to high freuqency with the advantageous mutation. Corssing-over can break apart associations between neutral variants and sweeping mutations, but obviously if a mutation is swept rapidly, there are fewer chances for crossing-over to occur. Because of this, selective sweeps generate troughs in genetic diversity, centred on the site of the advantageous mutation. The width and depth of the trough are dependant on the rate of crossing-over but also, crucially, the strength and frequency of advantageous mutations.

In wild mice, there are troughs in diversity surrounding functional elements. In a recent analysis, we estimated the frequnecy and selection coefficeients of advantageous mutations that occur in mice using distribution of derived allele frequencies (Booker and Keightley submitted). We showed that the parameters of selection obtained from the uSFS are unable to explain the patterns of selection observed in the genome.

Recently, we have estimated the DFE using the uSFS for wild mice and shown that the parameters of selection that we infer do not explain the reductions in diversity observed around protein-coding exons. 

In this study, we use a model of selective sweeps to estimate the strength and frequency of advantageosu mutations that occur within protein-coding exons and regulatory elements. The model we use incorporates the confounding effects of background selection as well as gene conversion as both processes are known to influence estimates of selection obtained from patterns of nucleotide polymorphism. We use simulations to validate our approach and to also demonstrate that uSFS-based methods fail to detect the strength and frequnecy of new mutations when they are rare.



\section*{Materials and Methods}

	\subsection*{Model of Recurrent Sweeps and Background Selection}
Background selection (BGS) is often modelled as the reduction in diversity experienced by a focal neutral site caused by deleterious mutations occurring at linked selected sites. An approximation for the reduction in diversity caused by background selection:
\begin{equation}
B = \frac{N_{e}}{N_{0}} \approx exp\Bigg[- \sum \limits_{x} \int \limits_{0}^1 \frac{u_{x}f_x(t)\mathop{dt}}{t \Big( 1 + \frac{(1-t)r_{x,y}}{t} \Big)^2} \Bigg]
\end{equation}
Where the sum is over all linked selected sites, the integral is over the distribution of fitness effects for deleterious mutations, \emph{$u_{x}$} is the deleterious mutation rate, \emph{t} is the reduction in fitness for heterozygotes (assumed to be \(\frac{s}{2}\) ), $r_{x,y}$  is the recombination distance between the focal neutral site and the selected site and fx(t) is the proportion of sites in the DFE with a selection coefficient of t.

	Background selection (BGS) and selective sweeps (SSWs) are processes that induce coalescence. If we assume that the two are independent exponential processes, then the rates at which they induce coalescence can simply be summed (KIM AND STEPHAN 2000). While this assumption has been shown to hold reasonably well, in reality BGS may influence the effects of selective sweeps and \emph{vice versa}
	The assumption that selective sweeps and background selection are independent has been made before (KIM AND STEPHAN 2000; CORBETT-DETIG et al. 2015; ELYASHIV et al. 2016; CAMPOS et al. 2017) but in reality, BGS may influence the fixation probabilities of new advantageous mutations (REF?). 

The model we use here is an extension to the model used by CAMPOS et al. (2017) suggested by Charlesworth (unpublished). 
\begin{equation}
\frac{\pi_{j}}{\pi_{0}} \approx  \frac{1}{B_{j}^{-1}  + B2N_eP_{sc,j}}
\end{equation}
	Where \(\frac{\pi_{j}}{\pi_{0}}\) is the reduction in neutral genetic diversity at site j relative to the expectation in the absence of selection at linked sites. The differences between our model and that used by Campos et al (2017) is that B is in the second term in the denominator of Equation 1. B is the reduction in pairwise coalescence times due to the effects of background selection which is calculated using Equation 1. Multiplying the rate of sweep induced coalescence ($P_{sj}$) by B reflects the reduction in fixation probability of new mutations caused by background selection. 

\begin{equation}
P_{sc,j} \approx V_a \tau\gamma_a^{\frac{-4r_{i,j}}{s}} 
\end{equation}
The term $V_{a}$ = $2\mu$ $p_{a}\gamma_{a}$ is the rate of sweeps per generation, where 𝜇 is the per-base pair per generation mutation rate (assumed to be $5.4 \times 10^{-9}$ (UCHIMURA et al. 2015)), $p_a$ is the fraction of new mutations occurring within a focal element that are advantageous and $\gamma_a$  is the scaled selection coefficient of a new mutation. It is straightforward to incorporate a distribution of advantageous mutation effects to Equation 3:
		\begin{equation}
P_{sc,j} \approx \int \limits_{0}^{1} f_x(\gamma)   V_a \tau\gamma_a^{\frac{-4r_{i,j}}{s}} \mathop{d\gamma}
\end{equation}	
In this study, we assume an exponential distribution for the distribution of fitness effects for advantageous mutations. 

	We estimate $\gamma_a$ and $p_a$ by fitting the relationship between nucleotide diversity and distance to functional elements by non-linear least squares using the \emph{lmfit} (0.9.7) package for Python 2.7.
 
	\subsection*{Simulations}
	We simulated background selection and selective sweeps using the forward-time simulation package SLiM (v1.8; Messer 2012). We performed simulations of a single 1Kbp protein-coding exon, flanked up and downstream by 70Kbp of strictly neutral sequence. 75\% of sites in the simulated exon were subject to selection (i.e. nonsynonymous sites) and the remainder were strictly neutral (i.e. synonymous sites). The population-scaled mutation rate (\emph{$4N_{e}\mu$}) was set to 0.01 and the population-scaled recombination rate (\emph{$4N_{e}r$}) was set to either 0.009, 0.0045 or 0.001. For a given distribution of fitness effects, see below, we performed 1,000 replicate simulations at each recombination rate resulting in 3,000 replicates per set of selection parameters. We ran simulations of 1,000 individuals for 20,000 generations to ensure that equilibrium conditions have been reached. At the final generation, 20 haploid genomes were sampled from the population. From these, we extracted the patterns of diversity around the exon or the uSFSs for nonsynonymous and synonymous sites within the exon itself.



\begin{table}
\centering

\caption{Distributions of fitness effects in simuations. In all simulations, deleterious mutations were drawn from an exponential DFE with $\gamma_d$ = -48.50. Where $\gamma = 2N_es$}
 \begin{tabular}{c c c l} 

  \hline
  DFE Model & $\gamma_a$ & $p_a$ & Label \\ [0.5ex] 
 \hline
 \multirow{3}{*}{Bimodal} & 400 / 20 & 0.001 / 0.009 & Bimodal \\
 						 & 400 / 20 & 0.0001 / 0.0009 & Bimodal - div10 \\
 						 & 400 / 20  & 0.00001 / 0.00009  & Bimodal - div100 \\
 \hdashline
 \multirow{4}{*}{Exponential} & 200 & 0.001 & Exp200 \\
 						  & 200 & 0.0001 & Exp200 - div10\\
 						  & 20 & 0.001 & Exp20 \\
 						  & 20 & 0.0001 & Exp20 - div10 \\
  \hdashline						 
  \multirow{4}{*}{Fixed} & 400 & 0.001 & Nes200 \\
 						  & 400 & 0.0001 & Nes200 - div10 \\
 						  & 20 & 0.001 & Nes10 \\
 						  & 20 & 0.0001 & Nes10 - div10 \\
\hline

\end{tabular}    
    \label{tab:castaneusEstimates}

\end{table}


	\subsection*{Analysis of the uSFS}

	We estimate the strength and frequency of new mutations using the uSFS. We analyse the uSFS using either the method of Schneider et al. (2011) as implemented in DFE-alpha, or the methods of Tataru et al. (2017) as implemented in polyDFE. Both methods estimate the rate and strength of advantageous mutations using the unfolded site frequency spectrum (uSFS). However, the models implemented in the two differ in their underlying assumptions. The Schneider et al. (2011) approach builds upon the Wright-Fisher transition matrix methods developed by Keightley and Eyre-Walker (2007) to estimate the distribution of fitness effects for harmful mutations. The methods implemented by Tataru et al. (2017) build upon Sawyer and Hartl’s Poisson random field model.
Throughout the rest of the paper, we will refer to these methods by the names of the programs in which they are implemented.

DFE-alpha does not currently allow the user to estimate an exponential distribution of advantageous mutational effects. We use the PRF-based method polyDFE to infer the parameters of the DFE from the uSFS of simulations incorporating the 

		\subsubsection*{1. DFE-alpha}
		The methods of Schneider et al (2011) are implemented in the program DFE-alpha. We analyse the simulation data using DFE-alpha

	\subsection*{Analysis of Mouse Data}

Halligan et al. (2013) sequenced the genomes of 10 wild-caught \emph{Mus musculus castaneus} individuals to high coverage using Illumina paired-end reads. We used the variants called in that study to obtain estimates nucleotide diversity.

From the edges of exons (CNEs), I extracted the SFS in windows of 1Kbp (100bp) extending to distances of 100 Kbp (5Kbp). All non-CpG sites in these windows were extracted, and mouse-rat divergence was calculated. Using either the LD-based map or the Cox-map I calculated the genetic distance between an analysis window and the centre of the focal element. 

	In mice, there is either non-crossover gene conversion, or gene conversion associated with crossing over events. It has been shown that the average gene conversion tract length differs in crossover or non-crossover gene conversion so we extended the recombination distance used 

	There are two types of genetic maps available for mice, those constructed by performing crosses and those inferred from patterns of linkage disequilibrium. The two maps will have benefits and drawbacks. Firstly, maps based on pedigree information are unbiased. They give a description of the locations and rates of crossing over events in the genome,. However, pedigree-based maps require a large number of individuals to be genotyped, which has meant that researchers have often been limited to using a relatively small number of genetic markers. Recombination maps based on linkage disequilibrium, on the other hand, use patterns of linkage disequilibrium to infer the populiaton-scaled recombination rates ($4N_er$) across the genome. LD-based approaches can provide inferences of recombination rates at very fine-scales across the genome, enabling researchers to locate recombination hotspots (reference to a review? – necessary?). A drawback of LD-based approaches is that the recombination rate estimates they produce are confounded with the level of genetic diversity, since both are functions of the effective population size ($N_e$). In this study, we incorporate genetic distances using both pedigree-based and LD-based recombination maps constructed for \emph{Mus musculus}. We use the (COX et al. 2009) genetic map, which was constructed with 10,195 SNPs genotyped in 3,546 meioses. 

	Rates of initiation of gene conversion in mice are known in mice.
A difficulty is that genetic 
 
 

\section*{Results}
\subsection*{Estimating selection parameters from simulations}


\subsection*{Estimates of selection for \textit{M. m. castaneus}}

	We estimated the parameters of a model of recurrent selective sweeps acting in two different classes of functional elements in \textit{M. m. castaneus}. We compared parameters obtained when incorporating gene conversion and background selection.
	
	Additionally, if selective sweeps occur frequently, the assumpiton that multiple sweeps do not interfere with each other may be violated. We found that correcting for multiple, competing sweeps did not make much difference on parameter estimates. This sugegsts that the rate of sweeps in mice is too low for the 



\begin{table}[h]
\caption{Parameters of positive selection in \textit{M. m. castaneus} estimated by fitting a model of selective sweeps to troughs in diversity around functional elements. Parameters were obtained assuming the gene conversion parameters estimated by \cite{RN263}. Standard errors are shown below in square brackets below point estimates}
\centering
	\label{EstimatesCastaneus}
        \begin{tabular}{cccccc}

        \hline
  %            & \multicolumn{2}{c}{} & \multicolumn{2}{c}{}  \\
  Element  &  $\gamma_{a,1}$ & $p_{a,1}$ &$\gamma_{a,2}$ & $p_{a,2}$ &  \\ [0.5ex] \hline

\multirow{2}{*}{Protein-Coding Exons} &  8,470 & 2.22 x $10^{-5}$ & 22.3 & 0.0202 & \multirow{4}{*}{\textit{castaneus} map}\\
   &  [ 672 ] & [ 2.21 x $10^{-6}$ ]& [ 3.39 ] & [ 4.38x $10^{-3}$ ] & \\ 
\multirow{2}{*}{Conserved Non-Coding Elements}  & 432 & 1.12 x $10^{-3}$ & 14.5 & 0.0298 & \\
  &   [ 21.2 ] & [ 8.86 x $10^{-5}$ ]& [ 3.17 ] & [ 8.22 x $10^{-3}$ ] &\\ \hdashline
\multirow{2}{*}{Protein-Coding Exons} &  4,100 & 2.45 x $10^{-5}$ & 117 & 6.16 x $10^{-4}$  & \multirow{4}{*}{\textit{Cox} map}\\
   &  [ 640 ] & [ 5.56 x $10^{-6}$ ]& [ 49.9 ] & [ 2.78 x $10^{-4}$ ] & \\ 
   
\multirow{2}{*}{Conserved Non-Coding Elements}  & 357  &  4.77 x $10^{-4}$ & 5.95 & 0.0454 & \\
  &   [ 46.8 ] & [ 9.37 x $10^{-5}$ ]& [ 3.533 ] & [  0.0358 ] &\\ \hdashline
        \end{tabular}
    \label{tab:Table2}
    
\end{table}

\linespread{1}



\section*{Discussion}

The rate of sweeps we estiamted for 

\subsection*{The contribution of adaptation in protein-coding and regulatory regions to phenotypic evolution in mice}

An enduring question in evolutionary biology has been the extent to which protein-coding and regulatroy regions of the gneome contribute to fitness change. In this study we have found that the strenght of selection acting on new mutations occuring in protein-coding regions far exceeds that of those occuring in regulatory regions (Table X). However, the total contribution that the two classes of sites make to fitness changes will depend upon the total number of the different site types in a species' genome. In mice, there is a much large number of non-coding sites in the genome inferred to be targets of selection than there are sites that encode proteins (Halligan et al 2013; MORE MORE MORE). If the strenth of selection on advantageous mutations occuring in conserved non-coding regions was similar to the strength of mutations occuring in Equation \ref{eq:fitness}


The rate of adaptive fitness change ($\Delta W$) generated by mutations occurring at a particular class of sites can be modelled as

\begin{equation}
\label{eq:fitness}
\Delta W \propto \mu_a n_a E(s_a^2),
\end{equation}

where $\mu_a$ is the rate of advantageous mutations occuring at a particular class of sites, \textit{$n_a$} is the total number of sites in the genome corresponding to that class and \textit{$s_a^2$} is the square of the selection coefficient for advantageous mutations (Halligan et al 2013). 



\begin{table}
\centering

\caption{Rough estimates of the changes in fitness caused by new mutations occuring in protein-coding exons and CNEs. Estimates obtained assuming an effective population size of 420,000 and a per base-pair per generation mutation rate of 5.4 x $10^{-9}$ (Uchimura \textit{et al.} 2015).}
 \begin{tabular}{c c c c c} 

  \hline
		& $\mu_a$ (x $10^{-10}$)& $n_a$ (x $10^6$)& $s_a^2$ & $\Delta W$ \\ [0.5ex] \hline
	Exons & & & &  \\
	CNEs  & & & &  \\ \hline
\end{tabular}    

\end{table}



A weakness of this study is that we did not estimate the strength of selection acting on the untranslated regions (UTRs) of protein-coding genes.

To our knowledge, there are currently no methods that estimate the DFE using the site frueqnecy spectrum expected under either backgorund selection or selective sweeps. Rather, nuisance parameters or demographic models are used to account for the conribution of selection at linked sites to the SFS while estimating the DFE. However, there is potentially information present in the uSFS that would be useul for estimating the fitness effects of new mutations. Approximations for the uSFS expected under both BGS and selective sweeps have been developed (REFS), so a potential avenue for further research would be to incorporate these for making inferences from population genetic data.


\end{document}